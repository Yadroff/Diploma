\subsection{Метрики}
Мы определили ожидаемое поведение системы. Теперь мы установим критерии для оценки систем, за которыми ведется наблюдение. Оптимальный метод оценки отдельных компонентов системы заключается в использовании метрик, так как они позволяют сравнивать разные состояния системы и выбирать наилучшее из них.

Метрики могут быть различными и зависят от специфики конкретной системы. Например, для оценки производительности может использоваться метрика времени ответа системы на запрос пользователя, а для оценки надежности - частота отказов системы.

Важно отметить, что выбор метрик должен основываться на целях и задачах проекта. Каждая система имеет свои особенности и требования к работе, поэтому необходимо определить наиболее значимые параметры для ее функционирования.

Кроме того, использование метрик позволяет проводить анализ данных о работе системы и делать выводы об эффективности ее работы. Это помогает оптимизировать процессы и повышать качество продукта.

В данной работе ключевую роль играют метрики нейросетевых моделей.

\subsubsection{Метрики нейросетевых моделей}

Метрики, применяемые к нейросетевым моделям, можно глобально разделить на:
\begin{itemize}
    \item Метрики качества модели - показатели, позволяющие оценивать эффективность работы модели;
    \item Метрики потерь - показатели, позволяющие оценивать степень ошибки или несоответствия между прогнозами модели и реальными данными.
\end{itemize}

В качетсве метрик качества модели мы выбрали следующие:

\begin{itemize}
    \item $Precision$ - точность обнаруженных объектов, указывающая, сколько обнаружений было выполнено правильно;
    \item $Recall$ - способность модели идентифицировать все экземпляры объектов на изображениях;
    \item $mAP50$ - средняя точность, рассчитанная при пересечении, превышает порог объединения ($IoU$), равный 0,50. Это показатель точности модели, учитывающий только "простые" обнаружения;
    \item $mAP50-95$ - среднее значение средней точности, рассчитанное при различных пороговых значениях долговых расписок, варьируется от $0,50$ до $0,95$. Это дает полное представление о работе модели при различных уровнях сложности обнаружения.
\end{itemize}

В качестве метрик потерь мы выступает метрика $cls loss$ ($Classification loss$) - потеря классификации, определяющая погрешность в прогнозируемых вероятностях классов для каждого объекта на изображении по сравнению с исходным значением. Меньшее значение этой метрики означает, что модель более точно предсказывает класс объектов