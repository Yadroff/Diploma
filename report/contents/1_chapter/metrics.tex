\subsection{Метрики}
Мы определили ожидаемое поведение системы. Теперь мы установим критерии для оценки систем, за которыми ведется наблюдение. В качестве оценки систем в данной работе используются метрики.
Использование метрик позволяет проводить анализ данных о работе системы и делать выводы об эффективности ее работы. Это помогает оптимизировать процессы и повышать качество продукта.

В данной работе ключевую роль играют метрики нейросетевых моделей. Поэтому будем отслеживать метрики, применяемые к нейросетям.


Метрики, применяемые к нейросетевым моделям, можно глобально разделить на:
\begin{itemize}
    \item Метрики качества модели~--- показатели, позволяющие оценивать эффективность работы модели;
    \item Метрики потерь~--- показатели, позволяющие оценивать степень ошибки или несоответствия между прогнозами модели и реальными данными.
\end{itemize}

В качестве метрик качества модели мы выбрали следующие:

\begin{itemize}
    \item $Precision$~--- точность обнаруженных объектов, указывающая, сколько обнаружений было выполнено правильно;
    \item $Recall$~--- способность модели идентифицировать все экземпляры объектов на изображениях;
    \item $mAP50$~--- средняя точность, рассчитанная при пересечении, превышает порог объединения ($IoU$), равный 0,50. Это показатель точности модели, учитывающий только <<простые>> обнаружения;
    \item $mAP50-95$~--- среднее значение средней точности, рассчитанное при различных пороговых значениях $IoU$, варьирующихся от $0,50$ до $0,95$. Оно дает полное представление о производительности модели на разных уровнях сложности обнаружения.
\end{itemize}

В качестве метрик потерь мы выступает метрика $\textit{cls loss}$ ($\textit{Classification loss}$)~--- потеря классификации, определяющая погрешность в прогнозируемых вероятностях классов для каждого объекта на изображении по сравнению с исходным значением. Меньшее значение этой метрики означает, что модель более точно предсказывает класс объектов.