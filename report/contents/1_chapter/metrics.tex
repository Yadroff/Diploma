\subsection{Метрики}
Итак, мы определили ожидаемое поведение системы. Теперь мы установим критерии для оценки систем, за которыми ведется наблюдение. Оптимальный метод оценки отдельных компонентов системы заключается в использовании метрик, так как они позволяют сравнивать разные состояния системы и выбирать наилучшее из них.

Метрики могут быть различными и зависят от специфики конкретной системы. Например, для оценки производительности может использоваться метрика времени ответа системы на запрос пользователя, а для оценки надежности - частота отказов системы.

Важно отметить, что выбор метрик должен основываться на целях и задачах проекта. Каждая система имеет свои особенности и требования к работе, поэтому необходимо определить наиболее значимые параметры для ее функционирования.

Кроме того, использование метрик позволяет проводить анализ данных о работе системы и делать выводы об эффективности ее работы. Это помогает оптимизировать процессы и повышать качество продукта.