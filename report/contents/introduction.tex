\introduction % Структурный элемент: ВВЕДЕНИЕ

В России на постоянной основе проводятся научные исследования во многих областях. Результаты этих исследований публикуются в виде научных статей, которые являются важным инструментом для распространения новых знаний и научных открытий.
Только в электронной версии научной библиотеки опубликовано 52573694 \cite{eLib} статей, и все они написаны с помощью \LaTeX ~--- мощного инструмента для верстки и оформления математических формул и научных текстов, который позволяет создавать качественные и профессионально оформленные статьи. 
Также с помощью \LaTeX \quad можно готовить конспекты к предметам, причем это может делать как преподаватель, так и студент.

Однако, набор даже простых формул в LaTeX для неподготовленного человека может оказаться достаточно сложным и трудоемким занятием.
Для примера возьмем формулу

\begin{equation}
    \label{formula_example}
        f(x,y,\alpha, \beta) = \frac{\sum \limits_{n=1}^{\infty} 
        A_n \cos \left( \frac{2 n \pi x}{\nu} \right)} {\prod \mathcal{F} {g(x,y)} } 
\end{equation}

На рисунке ~\ref{formula_listing} показан листинг \LaTeX \quad кода этой формулы:

\begin{figure}
    \begin{lstlisting}
        f(x,y,\alpha, \beta) = \frac{
            \sum \limits_{n=1}^{\infty} A_n \cos 
            \left( \frac{2 n \pi x}{\nu} \right)
            } {
                \prod \mathcal{F} {g(x,y)} 
            } 
    \end{lstlisting}
    \caption{Листинг формулы ~\ref{formula_example}}
    \label{formula_listing}
\end{figure}

Как мы видим, используется много специальных символов (например, символ суммы, произведения, а также дроби, скобки и пр.), которые необходимо знать или тратить время для их поиска на просторах Интернета. 
В любом случае, требуется каждый раз компилировать pdf-файл для просмотра и проверки результата, что требуется некоторого количества времени.

В настоящее время появляется все больше различных нейросетей (например, Гигачат, Yandex GPT, ChatGPT, stable diffusion, Midjonery и тд), в том числе преобразующие рукописный текст на изображении в машинный, а также способных генерировать готовый код. 
Поэтому появляется мотивация для автоматизиции процесса преобразования формул из чистового варианта на бумаге в готовый  \LaTeX --код.

Однако, мир не стоит на месте, и компания $Mathpix$ придумала свое решение \cite{mathpix} этой задачи, которое распространяется по платной подписке, что не удовлетворяет требованию доступности ПО.

Целью работы является разработака программного обеспечения, выполняющего распознавание научного текста и  генерацию готового LaTeX кода.

Для достижения поставленной цели в работе были решены следующие задачи:
\begin{enumerate}
    \item Задачка
    \item Задачка еще одна:
        \begin{enumerate}
            \item Подзадачка для задачки
        \end{enumerate}
\end{enumerate}

Для разработки программного обеспечения необходимо изучить технологии и методы, решающие постаавленные задачи. Работа основывается на следующих библиотеках, технологиях, алгоритмах:
\begin{enumerate}
    \item $Python$ является основным языком программирования, который использовался для решения задач;
    \item $Tensorflow$
    \item $openCV$ - библиотека для обработки изображения
    \item Что-то ещё
\end{enumerate}

Результататом работы является:
\begin{enumerate}
    \item Результат
\end{enumerate}