\introduction % Структурный элемент: ВВЕДЕНИЕ

В России на постоянной основе проводятся научные исследования во многих областях. Результаты этих исследований публикуются в виде научных статей, которые являются важным инструментом для распространения новых знаний и научных открытий.
Например, по данным "Научной электронной библиотеки"\;опубликовано 52573694 \cite{eLib} статей, и большинство из них написано с помощью \LaTeX ~--- мощного инструмента для верстки и оформления математических формул и научных текстов, который позволяет создавать качественные и профессионально оформленные статьи. 
Также с помощью \LaTeX\; можно готовить конспекты к предметам, причем это может делать как преподаватель, так и студент.

Так как \LaTeX ~--- система верстки со своим особенным синтаксисом, то набор даже простых формул в нем может оказаться сложным процессом. Покажем весь процесс написания \LaTeX--кода на следующем простом примере
\begin{equation}
    \label{formula_example}
        f(x,y,\alpha, \beta) = \frac{\sum \limits_{n=1}^{\infty} 
        A_n \cos \left( \frac{2 n \pi x}{\nu} \right)} {\prod \mathcal{F} {g(x,y)} } 
\end{equation}

На рисунке ~\ref{formula_listing} показан листинг \LaTeX--кода формулы \ref{formula_example}:

\begin{figure}
    \begin{lstlisting}[language={[LaTeX]Tex}]
        f(x,y,\alpha, \beta) = \frac{
            \sum \limits_{n=1}^{\infty} A_n \cos 
            \left( \frac{2 n \pi x}{\nu} \right)
            } {
                \prod \mathcal{F} {g(x,y)} 
            } 
    \end{lstlisting}
    \caption{Листинг формулы ~\ref{formula_example}}
    \label{formula_listing}
\end{figure}

На данном листинге мы наблюдаем большое разнообразие команд \LaTeX. 
Все команды в \LaTeX\; начинаются со знака '\textbackslash', аргументы команд должны заключаться в фигурные скобки '\{\}',
а опции команды --- в квадратные '[]'. Также на листинге приведен пример вложенности команд друг в друга. 
Данные особенности усложняют не только чтение, но и в первую очередь написание кода документа.

После написания исходного кода документа, необходимо произвести компиляцию в один из поддерживаемых форматов: $\textit{.pdf}$ и $\textit{.dvi}$. 
На данном этапе отсеиваются все синтаксические ошибки, компилятор укажет (не всегда однозначно и понятно) на них в логе. 
Однако, такие ошибки как пропущенный или неправильно написанный символ, выравнивание и пр. не являются синтаксическими, и обнаружить их можно лишь при просмотре готового документа. 
После исправления ошибки необходимо заново компилировать документ, что увеличивает время написания документа.

Автоматическая генерация \LaTeX\--кода позволит сократить количество компиляций и уменьшить допускаемые человеком ошибки в написании формул, что ускорит процесс верстки документов.

Также существует множество источников, оцифрованных в форматы $pdf$, $djvu$ и др. 
Хранение в отсканированном виде источников обладает недостатками, такими как невозможность поиска по тексту и быстрого перемещения по нему.
Возможность преобразовывать данные файлы в \LaTeX--код позволит получить данные источники в оцифрованном, а не отсканированном виде, что устранит приведенные недостатки.

Стоит отметить, что уже существуют решения, позволяющие выполнять генерацию \LaTeX--кода по изображению, например, $Mathpix$ \cite{mathpix}. Однако, данное решение обладает рядом недостатков:
\begin{itemize}
    \item распространение по платной подписке;
    \item отсутствие открытого исходного кода;
    \item ограниченная поддержка языков: основным поддерживаемым языком является английский.
\end{itemize}

В настоящее время появляется все больше различных нейросетей (например, Гигачат \cite{gigachat}, ChatGPT \cite{chat_gpt}, Midjourney \cite{midjourney} и тд). Некоторые из них обучены под задачи распознавания текста, в том числе математического. 
Также существуют модели, способные генерировать готовый код. 

На основании вышенаписанного возникает идея автоматизации процесса преобразования научного текста на изображении в готовый \LaTeX--код.


Целью работы является разработка платформы, выполняющей распознавание научного текста и  генерацию готового \LaTeX--кода.

Для достижения поставленной цели в работе были решены следующие задачи:
\begin{itemize}
    \item определение пользовательских сценариев;
    \item определение требований к платформе;
    \item проектирование архитектуры платформы;
    \item разработка сервиса для распознавания научного текста;
    \item разработка сервиса для синхронизации локальных и облачных пользовательских данных;
    \item разработка настольного приложения для взаимодействия пользователя с сервером;
    \item тестирование и анализ прототипа платформы.
\end{itemize}

Для разработки программного обеспечения необходимо изучить технологии и методы, решающие поставленные задачи. Работа основывается на следующих библиотеках, технологиях, алгоритмах:
\begin{itemize}
    \item $Python$ является основным языком программирования, который использовался для решения задач;
    \item $Tensorflow$ --- библиотека для запуска и обучения моделей;
    \item $YOLO$ --- модель, созданная для классификации объектов на изображении;
    \item $OpenCV$ --- библиотека для обработки изображения;
    \item $tkinter$ --- $Python$ библиотека для работы с $GUI$;
    \item $grpc$ --- фреймворк для удаленного вызова процедур;
    \item $\textit{Weights \& Biases}$ --- сервис для отслеживания метрик обучения.
\end{itemize}

Результатом работы является:
\begin{itemize}
    \item сервис для коррекции перспективы изображения;
    \item сервис для нахождения формул на изображении;
    \item дообученная модель $YOLO$, позволяющая находить формулы на изображении;
    \item сервис для преобразования найденных формул в \LaTeX-код;
    \item настольное приложение с авторизацией в $Google\; Drive$ и взаимодействием с удаленным сервером с помощью $grpc$.
\end{itemize}