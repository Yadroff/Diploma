\introduction % Структурный элемент: ВВЕДЕНИЕ

В России на постоянной основе проводятся научные исследования во многих областях. Результаты этих исследований публикуются в виде научных статей, которые являются важным инструментом для распространения новых знаний и научных открытий.
Например, по данным "Научной электронной библиотеки" опубликовано 52573694 \cite{eLib} статей, и все они написаны с помощью \LaTeX ~--- мощного инструмента для верстки и оформления математических формул и научных текстов, который позволяет создавать качественные и профессионально оформленные статьи. 
Также с помощью \LaTeX \quad можно готовить конспекты к предметам, причем это может делать как преподаватель, так и студент.

Так как \LaTeX ~--- система верстки со своим особенным синтаксисом, то набор даже простых формул в нем может оказаться сложным процессом.
Для примера возьмем формулу

\begin{equation}
    \label{formula_example}
        f(x,y,\alpha, \beta) = \frac{\sum \limits_{n=1}^{\infty} 
        A_n \cos \left( \frac{2 n \pi x}{\nu} \right)} {\prod \mathcal{F} {g(x,y)} } 
\end{equation}

На рисунке ~\ref{formula_listing} показан листинг \LaTeX \quad кода этой формулы:

\begin{figure}
    \begin{lstlisting}[language={[LaTeX]Tex}]
        f(x,y,\alpha, \beta) = \frac{
            \sum \limits_{n=1}^{\infty} A_n \cos 
            \left( \frac{2 n \pi x}{\nu} \right)
            } {
                \prod \mathcal{F} {g(x,y)} 
            } 
    \end{lstlisting}
    \caption{Листинг формулы ~\ref{formula_example}}
    \label{formula_listing}
\end{figure}

На данном листинге мы наблюдаем большое разнообразие комманд \LaTeX.
После написания исходного кода документа, необходимо произвести компиляцию в один из поддерживаемых форматов: $.pdf$ и $\textit{.dvi (device independent file format)}$. Автоматическая генерация \LaTeX кода позволит сократить количество компиляций, что ускорит процесс верстания документов.

Также существует множество источников, оцифрованных в форматы $pdf$, $djvu$ и др. Хранение в отсканированном виде источников обладает недостатками, такими как невозможность поиска по тексту, отсутствие оглавления и быстрого перемещения по по нему.
Возможность преобразовывать изображение в \LaTeX-код позволит получить данные источники в оцифрованном, а не отсканированном виде, что устранит приведенные недостатки.

Стоит отметить, что уже существуют решения, позволяющие выполнять генерацию \LaTeX\; кода по изображению, например, \cite{mathpix}. Однако, данное решение распространяется по платной подписке и не имеет открытного исходного кода.

В настоящее время появляется все больше различных нейросетей (например, Гигачат \cite{gigachat}, ChatGPT \cite{chat_gpt}, Midjourney \cite{midjourney} и тд). Некоторые из них обучены под задачи распознования текста, в том числе математического. Также существуют модели, способные генерировать готовый код. 
На основании вышенаписанного возникает идея автоматизиции процесса преобразования формул из чистового варианта на бумаге в готовый  \LaTeX --код.


Целью работы является разработка платформы, выполняющей распознавание научного текста и  генерацию готового \LaTeX-кода.

Для достижения поставленной цели в работе были решены следующие задачи:
\begin{itemize}
    \item Определены пользовательские сценарии;
    \item Определены требования к платформе;
    \item Спроектирована архитектура платформы;
    \item Разработан сервис для распознования научного текста;
    \item Разработан сервис для синхронизации файловой системы облачного и локального хранилищ;
    \item Разработано приложение для взаимодействия пользователя с сервером;
    \item Протестировать и проанализирован прототип платформы.
\end{itemize}

Для разработки программного обеспечения необходимо изучить технологии и методы, решающие постаавленные задачи. Работа основывается на следующих библиотеках, технологиях, алгоритмах:
\begin{itemize}
    \item $Python$ является основным языком программирования, который использовался для решения задач;
    \item $Tensorflow$ - библиотека для запуска и обучения моделей
    \item $YOLO$ - модель, созданная для классификации объектов на изображении
    \item $openCV$ - библиотека для обработки изображения
    \item $tkinter$ - $Python$ библиотека для работы с $GUI$
    \item $grpc$ - фреймворк для удаленного вызова процедур
\end{itemize}

Результататом работы является:
\begin{itemize}
    \item Сервис для коррекции перспективы изображения
    \item Сервис для нахождения формул на изоображении
    \item Дообученная модель $YOLO$, позволяющая находить формулы на изображении
    \item Сервис для преобразования найденных формул в \LaTeX-код
    \item $GUI$ приложение с авторизацией в $Google\; Drive$ и взаимодействием с удаленным сервером с помощью $grpc$
\end{itemize}