\introduction % Структурный элемент: ВВЕДЕНИЕ

В России на постоянной основе проводятся научные исследования во многих областях. Результаты этих исследований публикуются в виде научных статей, которые являются важным инструментом для распространения новых знаний и научных открытий.
Только в электронной версии научной библиотеки опубликовано 52573694 \cite{eLib} статей, и все они написаны с помощью \LaTeX ~--- мощного инструмента для верстки и оформления математических формул и научных текстов, который позволяет создавать качественные и профессионально оформленные статьи. 
Также с помощью \LaTeX \quad можно готовить конспекты к предметам, причем это может делать как преподаватель, так и студент.

Однако, набор даже простых формул в LaTeX для неподготовленного человека может оказаться достаточно сложным и трудоемким занятием.
Для примера возьмем формулу

\begin{equation}
    \label{formula_example}
        f(x,y,\alpha, \beta) = \frac{\sum \limits_{n=1}^{\infty} 
        A_n \cos \left( \frac{2 n \pi x}{\nu} \right)} {\prod \mathcal{F} {g(x,y)} } 
\end{equation}

На рисунке ~\ref{formula_listing} показан листинг \LaTeX \quad кода этой формулы:

\begin{figure}
    \begin{lstlisting}
        f(x,y,\alpha, \beta) = \frac{
            \sum \limits_{n=1}^{\infty} A_n \cos 
            \left( \frac{2 n \pi x}{\nu} \right)
            } {
                \prod \mathcal{F} {g(x,y)} 
            } 
    \end{lstlisting}
    \caption{Листинг формулы ~\ref{formula_example}}
    \label{formula_listing}
\end{figure}

Как мы видим, используется много специальных символов (например, символ суммы, произведения, а также дроби, скобки и пр.), которые необходимо знать или тратить время для их поиска на просторах Интернета. 
После написания \LaTeX --кода формулы необходимо скомпилировать в файл $pdf$ формата для проверки правильности написания. Автоматический набор формул позволит кратно сократить количество компиляций и ускорить написание \LaTeX\; документа. 

В настоящее время появляется все больше различных нейросетей (например, Гигачат, Yandex GPT, ChatGPT, stable diffusion, Midjourney и тд). Некоторые из них обучены под задачи распознования текста, в том числе математического. Также существуют модели, способные генерировать готовый код. 
На основании вышенаписанного возникает идея автоматизиции процесса преобразования формул из чистового варианта на бумаге в готовый  \LaTeX --код.

Также существует множество источников, оцифрованных в форматы $pdf$, $djvu$ и др. Возможность преобразовывать изображение в \LaTeX-код позволит получить данные источники в оцифрованном, а не отсканированном виде.

Целью работы является разработака прототипа платформы, выполняющего распознавание научного текста и  генерацию готового \LaTeX-кода.

Для достижения поставленной цели в работе были решены следующие задачи:
\begin{itemize}
    \item Определены пользовательские сценарии;
    \item Определены требования к платформе;
    \item Спроектирована архитектура платформы;
    \item Разработан сервис для распознования научного текста;
    \item Разработан сервис для синхронизации файловой системы облачного и локального хранилищ;
    \item Разработано приложение для взаимодействия пользователя с сервером;
    \item Протестировать и проанализирован прототип платформы.
\end{itemize}

Для разработки программного обеспечения необходимо изучить технологии и методы, решающие постаавленные задачи. Работа основывается на следующих библиотеках, технологиях, алгоритмах:
\begin{itemize}
    \item $Python$ является основным языком программирования, который использовался для решения задач;
    \item $Tensorflow$ - библиотека для запуска и обучения моделей
    \item $YOLO$ - модель, созданная для классификации объектов на изображении
    \item $openCV$ - библиотека для обработки изображения
    \item $tkinter$ - $Python$ библиотека для работы с $GUI$
    \item $grpc$ - фреймворк для удаленного вызова процедур
    \item Что-то еще
\end{itemize}

Результататом работы является:
\begin{itemize}
    \item Сервис для коррекции перспективы изображения
    \item Сервис для нахождения формул на изоображении
    \item Дообученная модель $YOLO$, позволяющая находить формулы на изображении
    \item Сервис для преобразования найденных формул в \LaTeX-код
    \item $GUI$ приложение с авторизацией в $Google\; Drive$ и взаимодействием с удаленным сервером с помощью $grpc$
\end{itemize}