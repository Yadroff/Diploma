\abstract

\keywords{ОБРАБОТКА ИЗОБРАЖЕНИЙ, ПРЕОБРАЗОВАНИЕ ТЕКСТА, МАШИННОЕ ОБУЧЕНИЕ, НЕЙРОННЫЕ СЕТИ}

Объектом исследования в данной работе является научный текст и способы его обработки.

Цель работы –-- разработка платформы, выполняющей распознавание научного текста и  генерацию готового \LaTeX--кода.

Основное содержание работы состояло в разработке алгоритма распознования формул на изображении научного текста и последующей генерации \LaTeX--кода данного текста.

Основными результатами работы, полученными в процессе разработки, является обученная для поиска формул на изображении нейросетевая модель, модуль для генерации \LaTeX--кода.

Результаты разработки предназначены для автоматического преобразования научного текста в \LaTeX--код.

Использование результатов данной работы позволяет сократить время верстки научного документа, а также оцифровывать отсканированные источники.