\conclusion

В результате выполнения выпускной квалификационной работы бакалавра была разработана система, выполняющая распознование научного текста и генерацию его \LaTeX\; кода. Были решены следующие задачи:

\begin{itemize}
    \item опредены пользовательские сценарии;
    \item определены требования к платформе;
    \item спроектирована архитектура платформы;
    \item спроектирован и разработан рабочий прототип платформы;
    \item произведено тестирование прототипа на реальном изображении.
\end{itemize}

Развитие данной платформы, спроектированной и реализованной в данной работе, может позволить перевести существующие отсканированные документы в оцифрованный вид. На данном этапе платформа не способна конкурировать с аналогичными решениями, например, $Mathpix$ \cite{mathpix}.

Для успешного конкурирования необходимо улучшить качество распознования формул, поддержать распознование изображений в тексте, добавить распознование рукописного текста, научиться распозновать структуру текста (заголовки, списки, выделение курсивом, полужирным и пр.).