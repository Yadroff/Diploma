\documentclass{article}
\usepackage[T2A]{fontenc} % Кодировка шрифта
\usepackage[utf8]{inputenc} % Кодировка ввода
\usepackage[english, russian]{babel} % Пакет для поддержки русского языка
\usepackage{amsmath} % Для использования математических формул
    
\begin{document}
Теория последовательностей

Теория последовательностей

1. Понятие предела последовательности. Говорят, что последовательность $x_1,x_2,\ldots,x_n, \ldots$ имеет своим пределом число $a$ (короче, сходится к $a$), т.е.

\[
\lim_{n \to \infty} x_n = a
\]

если для любого $\epsilon > 0$ существует число $N = N(\varepsilon)$ такое, что

$|x_n - a| < \varepsilon$ при $n > N$

В частности, $x_n$ называется бесконечно малой, если
\[
\lim_{n \to \infty} x_n = 0
\]

Последовательность, не имеющая предела, называется расходящейся.
    
\end{document}
